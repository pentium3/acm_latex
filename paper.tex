
\documentclass[sigconf,anonymous]{acmart}
%\documentclass[sigconf]{acmart}

\AtBeginDocument{%
  \providecommand\BibTeX{{%
    Bib\TeX}}}

% to be able to draw some self-contained figs
\usepackage{tikz}
\usepackage{amsmath}
% \usepackage{amssymb}
\usepackage{booktabs}
\usepackage{multirow}
\usepackage{graphicx}
\usepackage[export]{adjustbox}
\newcommand*\rot{\rotatebox{90}}
\usepackage{comment}

% custom paragraph style
\newcommand{\stitle}[1]{\vspace{1ex}\noindent\textbf{#1}}
\newcommand{\utitle}[1]{\vspace{1ex}\noindent\underline{\textbf{#1}}}

% system name placeholder
\usepackage{xspace}
\newcommand{\ours}{\emph{Gadget}\xspace}
\newcommand{\eg}{\emph{e.g.}}
\newcommand{\etc}{\emph{etc.}\xspace}

\usepackage{subcaption}
\usepackage{graphicx}
\usepackage{tabularx}
\usepackage{tcolorbox}

\usepackage{enumitem}
\setitemize{noitemsep,topsep=0pt,parsep=0pt,partopsep=0pt}

\newcounter{VasiaNOC}
\stepcounter{VasiaNOC}
% \newcommand{\vasia}[1]{}
\newcommand{\vasia}[1]{\textcolor{magenta}{\small \bf [Vasia\#\arabic{VasiaNOC}\stepcounter{VasiaNOC}: #1]}}

\newcommand{\revision}[1]{\textcolor{black}{#1}}

\usepackage[font=small]{caption}
\usepackage[font=footnotesize]{subcaption}

%\algnewcommand\algorithmicswitch{\textbf{switch}}
%\algnewcommand\algorithmiccase{\textbf{case}}


\newcounter{urlcolor}
\stepcounter{urlcolor}
\newcommand{\urlcolor}[1]{\textcolor[RGB]{190,0,200}{#1}}

\usepackage{url} \def\UrlBreaks{\do\A\do\B\do\C\do\D\do\E\do\F\do\G\do\H\do\I\do\J \do\K\do\L\do\M\do\N\do\O\do\P\do\Q\do\R\do\S\do\T\do\U\do\V \do\W\do\X\do\Y\do\Z\do\[\do\\\do\]\do\^\do\_\do\`\do\a\do\b \do\c\do\d\do\e\do\f\do\g\do\h\do\i\do\j\do\k\do\l\do\m\do\n \do\o\do\p\do\q\do\r\do\s\do\t\do\u\do\v\do\w\do\x\do\y\do\z \do\.\do\@\do\\\do\/\do\!\do\_\do\|\do\;\do\>\do\]\do\)\do\, \do\?\do\'\do+\do\=\do\#} 


\usepackage{amsthm}
\theoremstyle{definition}
\newtheorem{definition}{Definition}[section]

\setcopyright{acmcopyright}
\copyrightyear{2018}
\acmYear{2018}
\acmDOI{XXXXXXX.XXXXXXX}
\acmConference[Conference acronym 'XX]{Make sure to enter the correct conference title from your rights confirmation emai}{June 03--05, 2018}{Woodstock, NY}
\acmPrice{15.00}
\acmISBN{978-1-4503-XXXX-X/18/06}




\begin{document}

\title{The Name of the Title Is Hope}


\author{Ben Trovato}
\authornote{Both authors contributed equally to this research.}
\email{trovato@corporation.com}
\orcid{1234-5678-9012}
\author{G.K.M. Tobin}
\authornotemark[1]
\email{webmaster@marysville-ohio.com}
\affiliation{%
  \institution{Institute for Clarity in Documentation}
  \streetaddress{P.O. Box 1212}
  \city{Dublin}
  \state{Ohio}
  \country{USA}
  \postcode{43017-6221}
}

\author{Lars Th{\o}rv{\"a}ld}
\affiliation{%
  \institution{The Th{\o}rv{\"a}ld Group}
  \streetaddress{1 Th{\o}rv{\"a}ld Circle}
  \city{Hekla}
  \country{Iceland}}
\email{larst@affiliation.org}

\author{Valerie B\'eranger}
\affiliation{%
  \institution{Inria Paris-Rocquencourt}
  \city{Rocquencourt}
  \country{France}
}

\renewcommand{\shortauthors}{Trovato et al.}


\begin{abstract}
  A clear and well-documented \LaTeX\ document is presented as an
  article formatted for publication by ACM in a conference proceedings
  or journal publication. Based on the ``acmart'' document class, this
  article presents and explains many of the common variations, as well
  as many of the formatting elements an author may use in the
  preparation of the documentation of their work.
\end{abstract}

%%
%% The code below is generated by the tool at http://dl.acm.org/ccs.cfm.
%% Please copy and paste the code instead of the example below.
%%
\begin{CCSXML}
<ccs2012>
 <concept>
  <concept_id>10010520.10010553.10010562</concept_id>
  <concept_desc>Computer systems organization~Embedded systems</concept_desc>
  <concept_significance>500</concept_significance>
 </concept>
 <concept>
  <concept_id>10010520.10010575.10010755</concept_id>
  <concept_desc>Computer systems organization~Redundancy</concept_desc>
  <concept_significance>300</concept_significance>
 </concept>
 <concept>
  <concept_id>10010520.10010553.10010554</concept_id>
  <concept_desc>Computer systems organization~Robotics</concept_desc>
  <concept_significance>100</concept_significance>
 </concept>
 <concept>
  <concept_id>10003033.10003083.10003095</concept_id>
  <concept_desc>Networks~Network reliability</concept_desc>
  <concept_significance>100</concept_significance>
 </concept>
</ccs2012>
\end{CCSXML}

\ccsdesc[500]{Computer systems organization~Embedded systems}
\ccsdesc[300]{Computer systems organization~Redundancy}
\ccsdesc{Computer systems organization~Robotics}
\ccsdesc[100]{Networks~Network reliability}


\keywords{datasets, neural networks, gaze detection, text tagging}


%%
%% This command processes the author and affiliation and title
%% information and builds the first part of the formatted document.
\maketitle


%-------------------------------------------------------------------------------
\section{Introduction}\label{sec:intro}
%-------------------------------------------------------------------------------

Stream processing technology powers numerous business applications, including continuous analytics, monitoring, fraud detection, and online recommendations~\cite{dayarathna2018recent,carbone2020beyond}. All major cloud providers offer stream processing as a managed service~\cite{aws-kinesis,azure-streams,alibaba,google-dataflow} and many large companies have developed in-house streaming analytics platforms~\cite{chen2016realtime,mei2020turbine,kulkarni2015twitter,floratou2017dhalion,noghabi2017samza}. 



%-------------------------------------------------------------------------------
\section{Preliminaries}\label{sec:background}
In this section, we provide background on stream processing and clarify basic concepts that we use throughout the paper. 

\subsection{Streaming dataflow concepts}\label{sec:streaming-dataflows}

In the dataflow model~\cite{Akidau2015,carbone2020beyond}, a streaming computation is represented as a logical directed  graph $G=(V,E)$, where vertices in $V$ represent \textbf{operators} and edges in $E$ denote \textbf{data streams}. Upon deployment, the logical graph is translated to a physical execution plan, $G'=(V', E')$, which maps operators to provisioned workers, in practice, threads. We call vertices in $V'$ \textbf{tasks} or \textbf{instances} of a logical operator in $V$ and edges in $E'$ physical data channels.
Tasks are typically scheduled once and are long-running.
Each task is assigned to exactly one worker and each worker may execute one or more tasks of the same or different operators. The assignment is system-specific; it is computed at deployment time and remains static throughout job execution, unless a reconfiguration occurs. In a \textbf{data-parallel} execution, all tasks of an operator execute an identical logic on disjoint partitions of the input stream and they communicate  with tasks of upstream and downstream operators via  messages. 



\section{Acknowledgments}

Identification of funding sources and other support, and thanks to
individuals and groups that assisted in the research and the
preparation of the work should be included in an acknowledgment
section, which is placed just before the reference section in your
document.

This section has a special environment:
\begin{verbatim}
  \begin{acks}
  ...
  \end{acks}
\end{verbatim}
so that the information contained therein can be more easily collected
during the article metadata extraction phase, and to ensure
consistency in the spelling of the section heading.

Authors should not prepare this section as a numbered or unnumbered {\verb|\section|}; please use the ``{\verb|acks|}'' environment.


%%
%% The next two lines define the bibliography style to be used, and
%% the bibliography file.
\bibliographystyle{ACM-Reference-Format}
\bibliography{references}


\end{document}
\endinput
